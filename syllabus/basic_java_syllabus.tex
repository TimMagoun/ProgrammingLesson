\documentclass[11pt,fleqn]{article}

%% This first part is the document header, which you don't need to edit.
%% Scroll down to \begin{document}

\usepackage[latin1]{inputenc}
\usepackage{enumerate}
\usepackage[hang,flushmargin]{footmisc}
\usepackage{mdframed}
\usepackage{minted}
\usepackage{color}
\usepackage{datetime}
\usepackage{graphicx}
\graphicspath{ {../images/} }

\setlength{\oddsidemargin}{0px}
\setlength{\textwidth}{460px}
\setlength{\voffset}{-1.5cm}
\setlength{\textheight}{20cm}
\setlength{\parindent}{0px}
\setlength{\parskip}{10pt}

\newcommand{\mil}[2][java]{\mintinline{#1}|#2|}
%% This command allows quick use of \mintinline feature, default language is java.
%%
%% USAGE: \mil (optional)[<language>] {content}
%%
%% EXAMPLE: \mil[python]{if not x == 3}
%% 			\mil{if (x.equals(y)}

\begin{document}
\title{Basic Java Course Syllabus}%Insert Title here
\author{Tim Magoun and Aravind Koneru}
%\date{\textit{compiled on} \today \hspace{3mm} \begin{tiny}\currenttime\end{tiny}}
\date{\textit{Compiled on} \today \hspace{1mm} at \currenttime}
\maketitle

\begin{abstract}
In order to create a proficient programming sub-team, the new members must know how to program in Java, and become comfortable with the concept of inheritance. This will be accomplished through a series of Java courses instructed with the help of the various lesson plans and assessments included in this project. The \emph{Basic Java Course} will use four segments of two hours each in order to teach students, from the ground up, about programming in Java. Note that this course is not a substitution to a proper course in Java, but instead is a crash-course to prepare students for basic robot controlling code for FIRST\textsuperscript{\textregistered} Robotics Competition
\end{abstract}

\begin{center}
\textbf{Syllabus}
\end{center}

\begin{itemize}
\item Setup Eclipse
\item Structure of Programming
\item Primitive Types
\item Basic Operators
\item Arrays
\item Comparative Operators
\item Flow Control
\item Methods
\item Objects
\item Modifiers
\item Java Library Features
\item Inheritance
\end{itemize}

\newpage

\section*{Day 1}
\subsection*{Note to Instructor:} Bring a copy of both 32 bit and 64 bit Eclipse in case of slow or no internet.
\subsection*{Objective:} By the end of this lesson, the students will be able to perform basic calculations using Java's primitive types.
\subsection*{Prerequisites:} Working computer with wifi capabilities the authority to install software.
\subsection*{Install (or Update) Eclipse}
\begin{enumerate}
\item Go to \texttt{https://eclipse.org/downloads/eclipse-packages/}
\item Click on the corresponding installer, 32 bit or 64 bit (if you don't know the version of OS present, choose the 32 bit installer)
\item Download the installer to a known location (ex. Downloads or Desktop)
\item Execute the installer file
\item Select Eclipse IDE for Java Developers
\item Confirm install location and select preferred shortcut locations
\item Accept EULA
\item Bogosort the digits of $\pi$
\item Launch Eclipse Neon and set up preferences, line numbers are highly recommended
\end{enumerate}
\subsection*{Homework:} Write a line of code that will calculate from the right to left. ex \mil{int x = 4 + 5 * ( 6 - 7)}
\newpage

\section*{Day 2}
\subsection*{Note to Instructor:} It is essential that the students recognize the modular nature of comparison operators, as it is used very often in FRC programming.
\subsection*{Objective:} In this lesson, the students will learn about single dimensional arrays, and the various control flow statements that exists in Java. By the end of this lesson, the students will be able to recognize and analyze logical and bit-wise comparisons, and use those comparisons to create simple loops that will manipulate a single-dimensional array.
\subsection*{Prerequisites:} Knowledge of the primitive types and basic operators.  

\subsection*{Homework:} Write a program that will begin by automatically fill in an array of \mil{double} with multiples of $1.7$, and then traverse through the array to change all of the elements that are odd to two times the initial value. Print out those values in a single line.
\newpage
\section*{Day 3}
\subsection*{Note to Instructor:} This is the hardest portion of learning basic Java. Use examples to emphasize the nature of objects and how classes act as a blueprint while objects act like machines produced by those blueprints. Students must have a good understanding of Objects in order to understand the Command Based Programming used in FRC.
\subsection*{Objective:} In this lesson, students will dive into the world of Object-Oriented Programming (OOP). After this lesson, students will be able to write basic classes. They will also learn about modifiers such as \mil{final} and \mil{static}
\subsection*{Prerequisites:} Before starting this lesson, the students must be comfortable with the manipulation of variables and the various operators. They should also know the different ways in which \mil{for}, \mil{while}, \mil{if} etc. could influence the execution of the program.
\subsection*{Homework:} Write a class named Circle, and include a default constructor and a overloaded constructor for a \mil{double radius}. Write methods that will return the area and perimeter using a value of $\pi$ from the included Math class.

Write a Triangle class that takes in a \mil{int[][]} that stores the integer coordinates of the points on the triangle. Then write a method in that class that returns the perimeter, area, and for challenge, the centroid of the triangle.

Write a NumberProcessor class that will contain a set of \mil{static} methods and constants. 
\begin{enumerate}
\item Implement the algorithm in Day 2's homework. As \mil{public double[] oddDoubler{double[] input}}
\item Implement an algorithm that returns the sum of the digits of the input integer
\end{enumerate}

\newpage

\section*{Day 4}
\subsection*{Note to Instructor:} After this lesson, the students should understand the basic relation between super and sub classes. They should also learn to write subclasses and understand how command based programming uses super-classes to provide structure to the project.

\subsection*{Objective:} The students will be able to write their own subclass to a given superclass, and understand how to override methods. They will also learn how the different modifiers affect the visibilities of classes, members, and methods.

\subsection*{Prerequisites:} This lesson is an extension of Object-Oriented Programming. The students must be able to write classes before starting to learn about inheritance.

\subsection*{Homework:} Write a new class, SmartCircle, that extends the Circle class written in the previous lesson. Include the necessary constructors, and modify it so that the radius could only be an \mil{int}. Implement a variable within the class so that it keeps track of the number of circles created in the program, and create the necessary setter and getter methods for that member (Hint: use modifiers that makes the variable unchanged throughout instances).


\end{document}