\documentclass[11pt,fleqn]{article}

%% This first part is the document header, which you don't need to edit.
%% Scroll down to \begin{document}

\usepackage[latin1]{inputenc}
\usepackage{enumerate}
\usepackage[hang,flushmargin]{footmisc}
\usepackage{mdframed}
\usepackage{minted}
\usepackage{color}
\usepackage{datetime}

\setlength{\oddsidemargin}{0px}
\setlength{\textwidth}{460px}
\setlength{\voffset}{-1.5cm}
\setlength{\textheight}{20cm}
\setlength{\parindent}{0px}
\setlength{\parskip}{10pt}
\begin{document}

\newcommand{\mil}[1]{\mintinline{java}|#1|}


%% MINTED USAGE:
%% \inputminted[<options>]{<language>}{<filename>}

\title{Introduction to \bf{Java}}
\author{Tim Magoun and Aravind Koneru}
\date{\textit{compiled on} \today \hspace{3mm} \begin{tiny}\currenttime\end{tiny}}
\maketitle

This test will evaluate the familiarity of basic programming concepts as well as the knowledge of the
Java programming language, which is used as the programming language of numerous FIRST\textsuperscript\textregistered
robotics competitions.


\vspace{5mm}
The following topics will be on this test:
\begin{itemize}
\item Primitive Types and Operations (\mil{int, byte, boolean, etc.})
\item Modifiers (\mil{final, public, static, etc.})*
\item Comparison Operators (\mil{==, !=, >=,etc.})
\item Assignment operators (\mil{+=, *=, =, etc})
\item Flow Control (\mil{if, for, break, etc})
\item Methods and Parameters*
\item Single- and Multi-Dimensional Arrays
\item Object Oriented Programming*
\item Inheritance and Polymorphism*
\item Programming Habits and Conventions
\end{itemize}

* Starred items are extremely important in programming a robot
\vfill
\begin{center}
\textbf{DO NOT BEGIN UNTIL INSTRUCTED TO DO SO}
\end{center}

\newpage
\begin{center}
\begin{large}
Use this page for scratch work if desired
\vfill
Scratch work will not be graded
\end{large}
\end{center}
\newpage
\begin{center}
\begin{large}
PART ONE: Multiple Choice
\end{large}
\end{center}
\textit{Instructions: Choose the correct solution to the problem, there is only one correct answer for each problem.}

\begin{enumerate}
\item The size of a \mil{boolean} variable is
	\begin{enumerate}
	\item 1 byte
	\item 4 bytes
	\item 1 bit
	\item 16 bits
	\end{enumerate}
\item When adding an \mil{int} to a \mil{double}, the resulting variable will be
	\begin{enumerate}
	\item an \mil{int} with lower precision
	\item an \mil{int} with the same precision
	\item a \mil{double} with lower precision
	\item a \mil{double} with same precision
	\end{enumerate}
\item When the modifier \mil{private} is used, where could one could access the member?
	\begin{enumerate}
	\item Inside the same \mil{class}
	\item Inside the same \mil{package}
	\item Inside the same superclass
	\item Only the processor could access the member
	\end{enumerate}
\item When should one use the modifier \mil{static}?
	\begin{enumerate}
	\item When the member should not be modified
	\item When the member needs to be shared across all instances of the class
	\item When the member should not be accessed by the end-user
	\item When the member changes in value frequently
	\end{enumerate}
\item What data type does a conditional statement return?
	\begin{enumerate}
	\item \mil{int}
	\item \mil{boolean}
	\item \mil{boolean* pointer}
	\item conditional statements do not return any data type
	\end{enumerate}
\item What is the outcome when one executes the following code?
	\begin{minted}{java}
	public static void main(String[] args) {
		int x = 3;
		int []y= {3,4};
		if((x > (int) Math.PI) && (y[2] <= 3)) 
			System.out.print("True");
		System.out.print("False");
	}
	\end{minted}
	\begin{enumerate}
	\item True
	\item True False
	\item False
	\item Runtime Error: ArrayIndexOutOfBoundsException
	\end{enumerate}
\item What is the outcome when one executes the following code?
	\begin{minted}{java}
	public static void main(String[] args) {
		int x = 3;
		int []y= {3,4};
		if((x > Math.round(Math.PI)) & (y[2] <= 3)) 
			System.out.print("True");
		System.out.print("False");
	}
	\end{minted}
	\begin{enumerate}
	\item True
	\item True False
	\item False
	\item Runtime Error: ArrayIndexOutOfBoundsException
	\end{enumerate}
\item Which of the following is an equivalent statement for (x $\|$ y) $\&\&$ !x
	\begin{enumerate}
	\item y \&\& !x
	\item x $\|$ y
	\item !y
	\item y \&\& (y $\|$ x)
	\end{enumerate}
\newpage
\item The statement y $\|$ (3 * x) $ > $ 24 evaluates
	\begin{enumerate}
	\item type \mil{int}
	\item type \mil{double}
	\item type \mil{String}
	\item type \mil{boolean}
	\end{enumerate}
\item The output of the following annoying program is
	\inputminted{java}{../src/LoopTracing.java}
	\begin{enumerate}
	\item NullPointerException
	\item 0
	\item 20
	\item The program is an infinite loop
	\end{enumerate}
	\newpage
\item What will be printed if you run the following program?
	\inputminted{java}{../src/SwitchTracing.java}
	\begin{enumerate}
	\item Nine Four Nine One Four Nine 
	\item Nine Four One Nine Four One Nine Four One
	\item Nine Four One
	\item Would not compile because of incomplete for-loop
\end{enumerate}
\item How is an \mil{int} passed into a method?
	\begin{enumerate}
	\item By reference
	\item By value
	\item By object
	\item By pointer
	\end{enumerate}
\item What's the output of the following program?
	\inputminted{java}{../src/MethodTracing.java}
	\begin{enumerate}
	\item Hello World! 1
	\item Hello World! 10
	\item Changed! 10
	\item Changed! 1
	\end{enumerate}
\item Which keyword is used to determine a subclass relationship between two classes?
	\begin{enumerate}
	\item \mil{catch}
	\item \mil{continue}
	\item \mil{implements}
	\item \mil{extends}
	\end{enumerate}
\item The preferred method of documentation generation for Java is called
	\begin{enumerate}
	\item Javadoc
	\item Java Manual
	\item Comments
	\item Java Forums
	\end{enumerate}
\end{enumerate}

\vfill
\begin{center}
\textbf{CONTINUE TO THE NEXT PAGE}
\end{center}

\newpage

\begin{center}
\begin{large}
PART TWO: Open Ended Response
\end{large}
\end{center}
\textit{Instructions: Write the most efficient solution to the following methods. You will \textbf{not} be given any extra paper.}

\begin{enumerate}
\item Write a method that will return an array of $n$ length, filled with the decimal approximations of the sequence $\left[ \frac{1}{1}, \frac{1}{2}, \frac{1}{3}, \frac{1}{4}, \cdots, \frac{1}{n} \right]$ where $n$ is the integer parameter of the method.

\mint{java}|public static double[] fractionGenerator(int n){|
\vfill
\begin{center}
\textbf{DO NOT CONTINUE UNTIL INSTRUCTED TO DO SO}
\end{center}
\newpage
\item Write a method that will recursively determine if a word \emph{str} is a palindrome, where \emph{str} is a string parameter of the method.

\mint{java}|public static boolean palindromeChecker(String str){|
\vfill
\newpage
\item Given the following super class:
%Relative location dependent file
\inputminted[firstline =6]{java}{../src/Counter.java}
\newpage
Write a subclass named \emph{IntervalCounter} that is a subclass of \emph{Counter} and has an additional integer instance field called interval. 
\inputminted{java}{../src/IntervalCounter.java}
\newpage
\item \textit{Extra Credit: }Explain, to the best of your ability, the significance of each of the key words in the following iconic signature line and why they are necessary for the proper execution of a Java program.

\hspace{10mm} \mil{public static void main(String[] args)}

\end{enumerate}
\vfill
\begin{center}
\textbf{END OF EXAM}
\end{center}
\end{document}