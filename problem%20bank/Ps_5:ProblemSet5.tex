\documentclass[11pt,fleqn]{article}

%% This first part is the document header, which you don't need to edit.
%% Scroll down to \begin{document}

\usepackage[latin1]{inputenc}
\usepackage{enumerate}
\usepackage[hang,flushmargin]{footmisc}
\usepackage{mdframed}
\usepackage{minted}
\usepackage{color}
\usepackage{datetime}
\usepackage{graphicx}
%%Please place all images used in documents in the images folder
\graphicspath{ {../images/} }

\setlength{\oddsidemargin}{0px}
\setlength{\textwidth}{460px}
\setlength{\voffset}{-1.5cm}
\setlength{\textheight}{20cm}
\setlength{\parindent}{0px}
\setlength{\parskip}{10pt}

\newcommand{\mil}[2][java]{\mintinline{#1}|#2|}
%% This command allows quick use of \mintinline feature, default language is java.

%% USAGE: \mil (optional)[<language>] {content}

%% EXAMPLE: \mil[python]{if not x == 3}
%% 			\mil{if (x.equals(y)}

\begin{document}
\title{Problem Set 5}%Insert Title here
\author{Tim Magoun and Aravind Koneru}
\date{\textit{Compiled on} \today \hspace{1mm} at \currenttime}
\maketitle

\begin{center}
\textbf{Do these problems for additional practice and challenge}
\end{center}

Reminders:
\begin{itemize}
    \item
        Finish last week's Homework before approaching this week's problems.

    \item
        Submit these problems to Dr. Rogers as soon as possible

    \item
        Email the .java files as we discussed in class instead of copying the code into the email

    \item
        If you don't solve every problem, it's fine. Just send us what you did. That being said,
        please refrain from copying the answers from online. All the instructors are aware that many
        of the problems that we give you can be found on StackOverflow and other community sites AND
        we do notice when a solution has been plagiarized.

    \end{itemize}

\begin{enumerate}
    \item
        \textbf{Array Difference}: Given an integer array find the difference between the numbers even and odd indices in the arrays. For example, if I have an array, [ 1, 2, 3, 4 ], then this method will return 2 as the sum of 4 and 2 is 2 greater than the sum of 1 and 3. Try implementing this in the method shown below. This problem will give you a good practice in using loops.

        \inputminted{java}{diff.java}

    \item
        \textbf{Student}: This problems involves creating a class called Student with certain methods and attributes. The attributes that a Student Class should have are his age, his name, his grade, his id and his GPA. Student should also have a constructor that defines the attributes mentioned above, and an isOlder() method that compares two students. This problem allows you to understand Object Oriented Programming. If you need help, look back at the homework Date.

   \end{enumerate}

\end{document}
